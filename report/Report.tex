\documentclass[12pt, TexShade, letterpaper]{report}

\usepackage{import}
\import{}{preamble.tex}

\begin{document}

\begin{titlepage}
		\begin{center}
			\vspace*{0.5cm}

			\LARGE
			\textbf{Analysis of 2x2 Cross-Over Trials}
			
			\vspace{1cm}
			
			\textit{Jules Lanari-Collard}
			
			\vspace{1.2cm}
			
			\includegraphics[width=0.25\textwidth]{mcglogo.png}
			
			\Large
			MATH 410 Final Report
			
			\vspace{-5mm}
			McGill University
			
			\vspace{-5mm}
			Montr\'eal, Qu\'ebec, Canada
			
			\vspace{5mm}
			August 16, 2024
			\small
			\vspace{0.5cm}
			{\color{red} \hrule height 0.75mm}
			
			\vspace{0.2cm}
			
			Supervised by Dr. José Correa
		\end{center}
\end{titlepage}

\setlength{\voffset}{2cm}
\renewcommand{\chaptermark}[1]{%
	\markboth{\thechapter.\ #1}{}}
\chapter*{Abstract}\markboth{Abstract}{}
	\label{chap:engAbstract}
%	\addcontentsline{toc}{section}{\nameref{chap:engAbstract}}

 % Start of ToC, LoT, gls
	\tableofcontents\thispagestyle{plain}

 	\clearpage
	\pagenumbering{arabic} % restart page numbers at one, now in arabic style

	% start of mainmatter
\chapter{Introduction to Cross-over Trials}
\section{The Cross-over Design}
A cross-over trial is defined as a "trial in which subjects are given sequences of treatments with the object of studying differences between individual treatments" \cite{senn2002crossover}. This differs from the traditional parallel group trial, where subjects typically only receive one treatment throughout the trial.

The most common and simple cross-over design, is the 2x2 cross-over design, whereby there exist two treatments and two treatment groups. In this design, for example comparing two treatments \textit{A} and \textit{B}, subjects receive either \textit{A} or \textit{B} in the first period, and then the other treatment in the second period. When only investigating the effects of one treatment (as opposed to comparing two treatments), the other 'treatment' is a placebo.

Cross-over designs are typically used for testing treatments for ongoing or chronic diseases, where "there is no question of curing the underlying problem which has caused the illness but a hope of moderating its effects through treatment" \cite{senn2002crossover}. Examples of such conditions include asthma, rheumatism, epilepsy and migraines. Due to time constraints and the risk of drop-out or carryover, cross-over designs are more suitable to single-dose trials, as opposed to trials involving multiple doses over a period of time \cite{senn2002crossover}.

\subsection{Consequences}
A unique characteristic of the cross-over design is that only the \textit{order} of treatment is randomised, which has the following consequences \cite{piantadosi2005clinical}:
\begin{itemize}
    \item The validity of treatment comparison does not depend on randomisation.
    \item Randomisation does not guarantee an unbiased comparison of treatments.
    \item Treatment groups differ with respect to their recent exposure to potentially effective treatments.
\end{itemize}
In conclusion, the primary issue is that the comparability of treatments is not guaranteed by the structure of the trial alone, but instead depends on the treatments themselves \cite{piantadosi2005clinical}.

\section{Advantages}
The principal advantage of a cross-over design is that it can lead to significant savings in resources \cite{senn2002crossover}. Firstly, when compared to a parallel group trial, a cross-over design only requires half as many subjects to obtain the same number of observations per treatment.

Secondly, the data can be interpreted in terms of subject-level \textit{difference to control}, eliminating between-patient variation \cite{senn2002crossover}, which is usually greater than within-patient variation \cite{piantadosi2005clinical}. This reduces the number of observations (and hence subjects) required for the same precision in estimation. Furthermore, within-subject responses to treatments are usually positively correlated \cite{piantadosi2005clinical}, reducing the variance of the estimated treatment difference (from control) and increasing efficiency, as demonstrated below:

\begin{quote}
    \textit{Noticing that:}
\end{quote}
\begin{equation*}
    cov(\bar{Y}_A, \bar{Y}_B)
= \rho_{AB}sd(\bar{Y}_A)sd(\bar{Y}_B) = \rho_{AB} \cdot \frac{\sigma}{\sqrt{n}} \cdot \frac{\sigma}{\sqrt{n}} = \rho_{AB}\frac{\sigma^2}{n}
\end{equation*}
\begin{quote}
    \textit{Assuming constant variance in observations between patients and no period or carryover effects, we then have:}
\end{quote}
\begin{align*}
    var(\hat{\delta}_{AB}) &=
    \frac{\sigma^2}{n} + \frac{\sigma^2}{n} - 2cov(\bar{Y}_A, \bar{Y}_B) \\
    &= 2\frac{\sigma^2}{n}(1-\rho_{AB})
\end{align*}
\begin{quote}
    \textit{Where $\sigma^2$ is the observation variance, $\bar{Y}_A$ and $\bar{Y}_B$ are the average observations for treatments A and B respectively, $\rho_{AB}$ is the within-subject response correlation, and $\hat{\delta}_{AB} := \bar{Y}_A - \bar{Y}_B$.}

    Given that $\rho_{AB} = 0$ in parallel group trials, positive correlation of within-subject responses will make a crossover trial more efficient \cite{piantadosi2005clinical}.
\end{quote}

An oft-overlooked benefit to the crossover trial is improved recruitment and reduced spill-over rates \cite{piantadosi2005clinical}. Since all subjects are guaranteed to receive each treatment at least once, it can be easier to recruit participants. Furthermore, if a treatment is known to be desirable (e.g. exercise), parallel group trials can be subject to spill-over, where subjects in the control group voluntarily take the treatment. Cross-over trials are evidently less at risk of this, since all subjects know they will receive the treatment.

\section{Disadvantages}
Many of the aforementioned benefits of the crossover design come with drawbacks. For example, the longer trial and multiple treatments could be seen as an inconvenience to patients and analysis of results is often more difficult and complex. Also, the design cannot be used for infectious diseases, where either significant deterioration or improvement in condition can occur during treatment. Patient drop-out is also very harmful to cross-over trials, since observations for all treatments are required for each patient to analyse the results. In comparison, a parallel group trial can still recover some information after a patient drops out \cite{senn2002crossover}.

There is also risk of \textit{period-by-treatment interaction} complicating analysis, where the effect of the treatment is not constant over time \cite{senn2002crossover}. In other words, when the period in which the treatment is administered affects the effectiveness of the treatment.

\subsection{Carryover}
The most common example of a period-by-treatment interaction is carryover, defined as "the persistence of a treatment applied in one period in a subsequent period of treatment" \cite{senn2002crossover}. In a cross-over design, this occurs when at the beginning of the second treatment period, patients are not in the state they would have been in, had they not received treatment in the first period. This causes the effect of one treatment to be misinterpreted as the effect of both treatments combined, introducing bias to the treatment effect estimates.

Carryover is both difficult to test for and then to adjust for. Tests for carryover effects are difficult to interpret independently of the treatment effect, and including carryover parameters in the model introduces additional uncertainty and requires additional (usually unreasonable) assumptions \cite{senn2002crossover}.

\subsection{Wash-out Period}
Senn [2002] proposes that the most efficient way to deal with carryover is introducing a \textit{wash-out period} to the experiment design. A wash-out period is "a period in a trial during which the effect of a treatment given previously is believed to disappear" \cite{senn2002crossover}. After the wash-out period, we can assume that all measurements taken are no longer affected by the previous treatment, the consequence being that all conclusions become conditional on the absence of a carryover effect.

\chapter{Visualisation of Cross-over Data}
This chapter will use as an example data published by Senn and Auclair \cite{senn1990graphical}, on a study comparing two drugs: \textit{formoterol} (For) and \textit{salbutamol}. Both drugs are bronchodilators, so the observation measurement used was \textit{peak expiratory flow} (PEF), defined as "a simple measure of the maximal flow rate that can be achieved during forceful expiration following full inspiration" \cite{peakflowrate2023}. The study employed a 2x2 cross-over design on 13 pediatric patients, with 7 randomised to the 'For-Sal' sequence and 6 to the 'Sal-For' sequence. A subsample of the data, in a 'longer' format, is outlined in table \ref{crossoverDataLong}. This format is more useful when modelling the data; for plotting purposes we can wrangle the data into a 'wider' format, as shown in table \ref{crossoverDataWide}.

\begin{table}
\centering
\caption{Sample of For/Sal Crossover Trial Results}
\centering
\begin{tabular}[t]{l|l|l|l|r}
\hline
Sequence & Subject & Period & Treat & PEF\\
\hline
For-Sal & 1 & 1 & For & 310\\
\hline
For-Sal & 1 & 2 & Sal & 270\\
\hline
Sal-For & 2 & 1 & Sal & 370\\
\hline
Sal-For & 2 & 2 & For & 385\\
\hline
Sal-For & 3 & 1 & Sal & 310\\
\hline
Sal-For & 3 & 2 & For & 400\\
\hline
For-Sal & 4 & 1 & For & 310\\
\hline
For-Sal & 4 & 2 & Sal & 260\\
\hline
Sal-For & 5 & 1 & Sal & 380\\
\hline
Sal-For & 5 & 2 & For & 410\\
\hline
\end{tabular}
\end{table}


\chapter{Analysis of Cross-over Data}

	% Begin Bibliography
	{
	
	\bibliography{references}
	\bibliographystyle{ieeetr}
	
	}
	
\end{document}
