\documentclass{beamer}
\usetheme[block=fill]{metropolis}
\title{Design and Analysis of 2x2 Cross-Over Trials with Continuous Data}
\author{Jules Lanari-Collard}
\institute{McGill University}
\date{August 23, 2024}

\usepackage{booktabs}

\begin{document}

\frame{\titlepage}

\section{Introduction to Cross-Over Trials}
\begin{frame}{The Cross-Over Design}
\begin{definition}
    A cross-over trial is a ``trial in which subjects are given sequences of treatments with the object of studying differences between individual treatments” \cite{senn2002crossover}.
\end{definition}
\end{frame}

\begin{frame}{Randomisation}
    Only the \textit{order} of treatments is randomised:
    \begin{itemize}
        \item Validity of treatment comparison does not depend on randomisation.
        \item Randomisation does not guarantee unbiased comparison of treatments.
        \item Treatment groups differ with respect to their recent exposure to potentially effective treatments.
    \end{itemize}
    
    \begin{alertblock}{Fundamental Issue of Cross-Over Design}
        The comparability of treatments is not guaranteed by the structure of the trial alone, but instead depends on the treatments themselves \cite{piantadosi2005clinical}.
    \end{alertblock}
\end{frame}

\begin{frame}{Advantages}
    \begin{itemize}
        \item More observations per treatment \cite{senn2002crossover}
        \item Data in terms of \textit{difference to control}
        \item Improved recruitment rates
        \item Reduced spill-over rates \cite{piantadosi2005clinical}
    \end{itemize}
\end{frame}

\begin{frame}{Disadvantages}
    \begin{itemize}
        \item Longer/inconvenient for subjects
        \item Complex analysis
        \item Cannot be used for infectious diseases
        \item Risk of drop-out
        \item Period-by-treatment interactions
    \end{itemize}
\end{frame}

\begin{frame}{Carryover}
    \begin{definition}
        Carryover is the persistence of a treatment applied in one period in a subsequent period of treatment \cite{senn2002crossover}.
    \end{definition}
    \begin{itemize}
        \item Introduces bias to direct treatment effect estimates.
        \item Difficult to test and adjust for.
        \item Best solution is to introduce a wash-out period \cite{senn2002crossover}.
    \end{itemize}
\end{frame}

\section{Summary and Analysis of Cross-Over Trial Data}

\begin{frame}{Athsma Trial}
    \begin{itemize}
        \item 2x2 cross-over design.
        \item Comparing effectiveness of Salbutamol (Sal) vs Formoterol (For) on children suffering from asthma.
        \item 13 patients randomised into either For-Sal or Sal-For sequence.
        \item Outcome measurement is \textit{peak expiratory flow rate} (PEFR), measured 8hrs after treatment.
        \item 1 day wash-out period
    \end{itemize}
\end{frame}

\begin{frame}{Sample Data}
    
-
/
                                                                                                   

/
                                                                                                   

-
                                                                                                   
 & 1 & Sal & 310\\
Sal-For & 3 & 2 & For & 400\\
\bottomrule
\end{tabular}
\end{table}

\end{frame}

\subsection{Summary Tables and Plots}

\begin{frame}{Summary Table}
\end{frame}

\subsection{Analysis}
\subsubsection{Matched-Pairs $t$-Test}
\begin{frame}{Test}
    
\end{frame}

\begin{frame}{Assumptions}
    
\end{frame}
\subsubsection{Mixed Model for Cross-Over Design}

\subsubsection{Controlling for Baseline Measurements}

\begin{frame}{Bibliography}
    \bibliography{report/references}
    \bibliographystyle{ieeetr}
\end{frame}

\end{document}